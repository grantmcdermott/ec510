\documentclass[12]{article}
%\input{/home/grant/Dropbox/LaTeX/preamble} %% Rather use self-contained preamble below

%% LAYOUT AND TITLES
\usepackage{setspace}
\onehalfspacing
\usepackage[margin=1.1in]{geometry}
\setlength{\parindent}{0pt}
\setlength{\parskip}{10pt}
\usepackage{titling}
\newcommand{\subtitle}[1]{%
	\posttitle{%
		\par\end{center}
	\begin{center}\large#1\end{center}
	\vskip0.5em}%
}
%% Change title format to be more compact
%\usepackage{titling}
%\setlength{\droptitle}{-2em}
%  \title{Syllabus}
%  \pretitle{\vspace{\droptitle}\centering\huge}
%  \posttitle{\par}
%  \author{Grant R. McDermott}
%  \preauthor{\centering\large\emph}
%  \postauthor{\par}
%  \predate{\centering\large\emph}
%  \postdate{\par}
%  \date{}

%% FONTS
\usepackage[normalem]{ulem} %% For strikeout font: \sout()
\usepackage{lmodern}
\usepackage{amssymb, amsmath}
\usepackage{fontspec}
% % See: https://tex.stackexchange.com/a/50593
\setmainfont[]{Fira Sans}
\setsansfont[]{Fira Sans}
\setmonofont[]{Fira Mono}
% \setmonofont[Mapping=tex-text]{inconsolata}	
\defaultfontfeatures{
    Path = /usr/share/texmf-dist/fonts/opentype/public/fontawesome/ }
\usepackage{fontawesome} % Ditto

%% MISC
\usepackage[colorlinks = true,
linkcolor = black,
urlcolor  = blue,
citecolor = blue,
anchorcolor = black]{hyperref}
\usepackage{tabularx}
\usepackage{booktabs}


\begin{document}

\title{Big data in economics (EC 410/510)}
\subtitle{\textsc{Spring 2022 syllabus}\vspace{-2ex}}
\author{Grant R. McDermott\\ Dept. of Economics, University of Oregon}
%\date{}  % Toggle commenting to test
\date{\vspace{-5ex}}
	
\maketitle
\section*{Summary}

\begin{tabular}{ll|l} 
	\textbf{When:} & \multicolumn{2}{l}{Mon \& Wed, 10:00--11:20} \\
	\textbf{Where:} & \multicolumn{2}{l}{140 TYKE} \\
	\textbf{Web:} & \multicolumn{2}{l}{\href{https://github.com/orgs/uo-510-2022s}{https://github.com/orgs/uo-510-2022s} (quarter)} \\
	& \multicolumn{2}{l}{\href{https://github.com/uo-ec607}{https://github.com/uo-ec607} (lectures)} \\
	\textbf{Who:} & Grant McDermott (instructor) & Boyoon Chang (GE) \\
	& \, \faMortarBoard \, Assistant Professor of Economics & \, \faMortarBoard \, Doctoral student in economics \\
	& \, \faEnvelopeO \, \href{mailto:grantmcd@uoregon.edu}{grantmcd@uoregon.edu} & \, \faEnvelopeO \, \href{mailto:bchang@uoregon.edu}{bchang@uoregon.edu} \\
	& \, \faHourglassHalf \, By appointment & \, \faHourglassHalf \, By appointment \\
\end{tabular} 

\section*{Course description}

Data are getting bigger. As data get big (i.e. they cannot fit in your
computer's memory) the conventional empirical tools of the applied economist's
toolbox often become inefficient or even ineffective. 
% Situations with big data abound in environmental, resource, and energy
% economics --- e.g., climate data, satellite imagery, high-resolution billing 
% data, and many transportation datasets.  
In this course, we will introduce, discuss, and implement some of the
key tools that have been developed to overcome these challenges. We will also
see how these ``big data'' tools be profitably repurposed for ``medium data''
settings too.  We will start by laying the foundations for effective data
science work, covering topics like version control and the shell (i.e. command
line). From there, we will learn how to handle data and become efficient
programmers in \textit{R} (our primary computational environment). While our
immediate focus will be on making the most of the local resources at our
disposal, the skills that we master here will serve us well once we scale up to
dedicated big data environments in the final section of the course. By the end
of the quarter, you will have connected to cloud-based based services and
high-performance computing clusters, queried petabyte-sized databases, and run
distributed code across a network of computers. More importantly, you will have
a better understanding of how computers work, what tools are at your disposal
for tackling big data problems, and how to meaningfully integrate them into your
everyday workflow.

\section*{Practical matters}

\subsection*{Class rules}

\textbf{Important:} We will be using a ``flipped'' classroom environment for
this course. In other words, the bulk of the lecture content for the course will
be delivered \textit{asynchronously} via pre-recorded videos. My expectation is
that you will watch and work through the videos on your own before we meet for
class. Having you watch the videos beforehand yields several benefits in a
course like this. It will save me (and you) from having to troubleshoot coding
problems live during lectures, it will allow you to rewatch material that you
didn't quite get upon first viewing, and it will generally free up time for
discussion in an otherwise very coding-centric course. We'll reserve actual
class time for two things: 1) troubleshooting and follow-up from the lecture 
videos, and 2) student presentations. Please note that you should attend the 
in-class sessions even though you will have watched the videos beforehand.

(Some lectures will still be ``live'' in the regular sense. Our first lecture
will definitely be one of these. I'll let you know ahead of time if any others
fall into this category too.)

\subsection*{Software requirements}

All of the software requirements for this course are free and open-source.
Please aim to have everything installed by the start of our first lecture. I
will be available for installation troubleshooting during the first week of the
quarter. If you want a detailed tutorial on how to achieve a perfect working
setup, I can think of no finer guide than Jenny Bryan \textit{et al}.'s
\url{http://happygitwithr.com/} (see esp. chapters 4 -- 11).

\vspace{-0.25cm}
\subsubsection*{\textit{R} and RStudio}

We will mainly be using the statistical programming language \textbf{\textit{R}}
(download \href{https://www.r-project.org/}{here}).\footnote{Mac users may
consider installing R through Homebrew. That's fine, but please use \texttt{`brew
install --cask r`} to do so (exactly as written here)... unless you're using one 
of the new M1-compatible machines. In the latter case, please speak to me about
installation options first.} 
Please make sure that you install the \textbf{RStudio IDE} too (download
\href{https://www.rstudio.com/products/rstudio/download/preview/}{here}).

\vspace{-0.25cm}
\subsubsection*{Git and GitHub Classroom}

We will also make extensive use of the \textbf{Git} version control system
(follow the OS-specific installation instructions
\href{http://happygitwithr.com/install-git.html}{here}). Once you have installed
Git, please create an account on \textbf{GitHub}
(\href{https://github.com/join}{here}) and register for an education discount to
get unlimited private repos
(\href{https://education.github.com/discount_requests/new}{here}).\footnote{GitHub
\href{https://blog.github.com/changelog/2019-01-08-pricing-changes/}{offers}
unlimited free private repos for everyone. However, you are limited to three
collaborators per private repo, so the education discount still makes sense.}
Now is probably a good time to tell you that I am going to run the course
through \href{https://classroom.github.com/}{GitHub Classroom}. You will receive
an email invitation to the course repo with instructions in due time, but
suffice it to say that this is how we'll submit assignments, provide feedback,
receive grades, etc.

\vspace{-0.25cm}
\subsubsection*{Other}

You are ready to start this course once you have installed R, RStudio, and Git
(as well as created an account on GitHub). The last thing I want you to do for
now is make sure that your system is configured to handle some additional


\begin{itemize}
	\item \textbf{Linux:} You should be good to go.  
	\item \textbf{Mac:} Install	the \href{https://brew.sh/}{Homebrew} package
	manager. I also recommend some additional software, depending on whether
	you're using one of the new M1-compatible machines or not:
	\begin{itemize}
		\item Non-M1 (older): Install the
		\href{https://github.com/rmacoslib/r-macos-rtools#installer-package-for-macos-r-toolchain-}{\emph{R}
		toolchain for MacOS}. This will automatically take care of several steps
		like installing Xcode, etc.
		\item M1 (newer): While the new M1 chips offer incredible performance,
		unfortunately they require several finicky steps to play nicely with
		some of the other tools that we'll be using. Please install Xcode by
		opening your Terminal and running \texttt{`xcode-select --install`}.
		There are several more steps that you'll need to take, but we'll save
		that for
		class or office hours.\footnote{\href{https://pat-s.me/transitioning-from-x86-to-arm64-on-macos-experiences-of-an-r-user/}{Here}
		is a thorough discussion if you want to go through them by yourself.}
	\end{itemize}
	\item \textbf{Windows:} Install \href{https://cran.r-project.org/bin/windows/Rtools/}{Rtools}. While its not essential, I also recommend that you install the \href{https://chocolatey.org/}{Chocolatey} package manager for Windows.
\end{itemize}

I will provide instructions for any further software requirements as the need
arises; i.e. when we get to the relevant lecture. On that note, the lectures
have all been posted ahead of time on the
\href{https://github.com/uo-ec607}{course website}. Each lecture lists all of
the \textit{R} packages and external libraries (if relevant) required for a
particular class. I'll try to remind you, but my expectation is that you will
look at these requirements and ensure that you have them installed
\textit{before} we start class. 

\subsection*{Textbook and other readings}

There's no set textbook for this course (Ed Rubin and I are
\href{https://grantmcdermott.com/ds4e}{working} on one.). The lecture notes are
very detailed and are thus ``self-contained''. However, I've drawn inspiration
from various sources; a few of which are listed below. You don't \textit{need}
to buy or read any of these (excellent) books to complete the course. But I can
eagerly recommend leafing through at least one or two of them. Each of these
books is freely available online if you can't afford a hard copy:
%
\begin{itemize}
	\item ``\href{http://r4ds.had.co.nz}{\textbf{\textit{R} for Data Science}}'' (Garrett Grolemund and Hadley Wickham)\footnote{FWIW, Jake VanderPlas's ``\href{https://jakevdp.github.io/PythonDataScienceHandbook/}{\textbf{Python Data Science Handbook}}'' is excellent option for anyone looking for a Python equivalent.}
	\item ``\href{http://socviz.co/}{\textbf{Data Visualization: A practical introduction}}'' (Kieran Healy)
    \item ``\href{https://adv-r.hadley.nz/}{\textbf{Advanced \textit{R}}}'' (Hadley Wickham)
    \item ``\href{https://geocompr.robinlovelace.net/}{\textbf{Geocomputation with \textit{R}}}'' (Robin Lovelace, Jakub Nowosad and Jannes Muenchow)
%    \item ``\href{https://keen-swartz-3146c4.netlify.app/}{\textbf{Spatial Data Science}}'' (Edzer Pebesma and Roger Bivand)
    \item ``\href{https://statlearning.com}{\textbf{An Introduction to Statistical Learning}}'' (Gareth James, Daniela Witten, Trevor Hastie, and Robert Tibshirani)
    \item Etc.
\end{itemize}

%\newpage
\section*{Evaluation and grading}

\subsection*{Grade determination}

%Grades will be determined according to a mix of regular assignments and
%in-class presentations. You will also be expected to evaluate each others' code
%and provide constructive feedback for improvements. There will be no final
%exam, although you may be asked to give a final presentation on a topic TBD.

Grades will be determined as follows:

\begin{table}[!h] \centering 
	%\caption{\textsc{grades} }
	\label{tab:grades} 
	\begin{tabularx}{0.9\textwidth}{Xr|Xr} 
		\toprule
		\multicolumn{2}{c}{EC 410}	& \multicolumn{2}{c}{EC 510}  \\
		\midrule
		1 $\times$ Short presentation  			&	 5\%	& 1 $\times$ Short presentation			& 5\% \\
		1 $\times$ Long presentation  			&	 10\%	& 1 $\times$ Long presentation			& 10\% \\
		1 $\times$ Intro HW assignment			&	 10\%	& 1 $\times$ Intro HW assignment		& 5\% \\
		3 $\times$ HW assignments (25\% each)	&	 75\%	& 3 $\times$ HW assignments (20\% each)	& 60\% \\
												& 			& OSS contribution						& 20\% \\
		\bottomrule
		% \multicolumn{4}{l}{\footnotesize Note: A class participation bonus worth an additional 2.5\% will be awarded at my discretion.} \\
	\end{tabularx} 
\end{table} 

You will only be graded on the items that are listed above. No additional work
or submissions will be considered. There is no final exam or project for this
course. Any changes or specific requirements will be made clear as we proceed
through the course. In the meantime, here are some additional details.

\vspace{-0.25cm}
\subsubsection*{Homework assignments}

With the exception of the first (short) introductory assignment, your homework
assignments are to be completed in \textbf{teams of two}. As we'll see, these
assignments will all be distributed and graded through GitHub Classroom. Links
will be provided and I'll cover the details in class. Please note that late
submissions will not be graded. \vspace{-0.25cm}

\subsubsection*{Presentations}

Each person in the class will be expected to give two presentations.  Both types
are first-come, first-served.
\begin{itemize}
	\item ``Short''. A 2--5 minute, show-and-tell talk on a ``base'' R function.
	I have a list of suggested options, but really this is just for you to
	explore and share some fun things about the language. Base R has many hidden
	(underrated) gems that will make you a more productive and capable
	programmer in the long-run, even if you	stick with user-written packages
	most of the time.  \item ``Long''. A 10 minute talk accompanied by slides
	(which you'll share with the rest of the class). You'll choose from a list
	of prescribed presentation topics---corresponding to the lecture topic at
	hand---that I shall provide in due course.
\end{itemize}
In general, your presentations will be graded on how clearly you communicate 
the topic at hand. The primary question that I'll be asking myself when grading 
is: ``Could other students easily understand the concept(s) based on your
talk?''

\vspace{-0.25cm}
\subsubsection*{OSS contribution (EC 510 only)}

You are going to contribute to open-source software (OSS) in some way, shape, or
form. This could be by identifying and correcting bugs in a package that you
use. Or, it could be by contributing material (e.g. documentation) to an
open-source project. I particularly want to encourage you to contribute to the
Library of Statistical Techniques (\url{https://lost-stats.github.io/}). There's
clearly quite a bit of leeway here and I'll need to sign off on whatever you
propose. Similarly, depending on the scope and size, you may need to make
several different contributions to fulfill the requirement.

\subsection*{Honesty and academic integrity}

Students caught cheating or plagiarizing will automatically be assigned a zero
grade. Please acquaint yourself with the Student Conduct Code at
\url{http://studentlife.uoregon.edu}. More important for this class, individual 
team member will be expected to pull their weight for your paired
assignments.\footnote{You'll see that we can actually track individual member
contributions, although I really hope it doesn't come to that.}

\subsection*{Accessibility}

If you have a documented disability and anticipate needing accommodations in
this course, please make arrangements with me during the first week of the term.
Please also request that the \href{https://aec.uoregon.edu/}{Accessible
Education Center} send me a letter verifying your disability. Students with
infants or young children that need ongoing care should similarly come and see
to me. We'll have to take it on a case-by-case basis, but I'll do my utmost to
accommodate you.

\newpage
\section*{Tentative lecture outline}
\label{sec:outline}

\textit{Note: We only have 80 minutes allocated for each lecture. I expect that
several individual topics will run over two or more lecture slots. Please bear
that in mind as you look over this tentative outline.}

\subsection*{Foundations}

\textit{Expected no. of lectures slots: 5 (6)}

\begin{itemize}
	\item Introduction: Motivation, software installation, and data visualization
	\item Version control with Git(Hub)
	\item Learning to love the shell
	\item \textit{R} language basics (\textit{optional})
\end{itemize}

\subsection*{Data wrangling, I/O, and acquisition}

\textit{Expected no. of lectures slots: 5}

\begin{itemize}
	\item Data cleaning and wrangling: (1) Tidyverse
	\item Data cleaning and wrangling: (2) data.table
	\item Big data I/O
	\item Webscraping: (1) Server-side and CSS
	\item Webscraping: (2) Client-side and APIs
\end{itemize}


\subsection*{Programming}

\textit{Expected no. of lectures slots: 4}

\begin{itemize}
	\item Functions in \textit{R}: (1) Introductory concepts
	\item Functions in \textit{R}: (2) Advanced concepts
	\item Parallel programming
\end{itemize}

\subsection*{Cloud resources and distributed computation}

\textit{Expected no. of lectures slots: 6}

\begin{itemize}
	\item Docker
	\item Cloud computation (Google Compute Engine)
	\item High performance computing (UO Talapas cluster) 
	\item Databases
	\item Spark
\end{itemize}


\subsection*{Other potential topics (time permitting)}

\begin{itemize}
	\item Regression tools for big data problems
	\item Spatial analysis
	\item Networks
	\item Deep learning
	\item Automation and workflow
	\item Rcpp (i.e. integrating C++ with R)
\end{itemize}

\end{document}
